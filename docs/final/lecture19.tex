% Generated by my modifications to the default Pandoc beamer template.
% Karl Browman has a tutorial about how to make Beamer sane:
% http://kbroman.wordpress.com/2013/10/07/better-looking-latexbeamer-slides/

% 12pt by default; handout if the handout template variable is defined
%\documentclass[12pt,ignorenonframetext,handout,12pt]{beamer}
\documentclass[12pt,handout]{beamer}
\usetheme{metropolis}

\providecommand{\tightlist}{%
  \setlength{\itemsep}{0pt}\setlength{\parskip}{0pt}}


% theme, colortheme, and fonttheme with sensible defaults
%%\usetheme{metropolis}
%%%\usecolortheme{dove}
%%
% if it is a handout show the notes text; otherwise don't
\setbeameroption{hide notes}
\setbeamertemplate{note page}[plain]

% Don't show things we don't want to see
\beamertemplatenavigationsymbolsempty
\hypersetup{pdfpagemode=UseNone} % don't show bookmarks on initial view

% Slide number in lower right
\definecolor{gray}{RGB}{155,155,155}
\setbeamertemplate{footline}{%
    \raisebox{5pt}{\makebox[\paperwidth]{\hfill\makebox[20pt]{\color{gray}
          \scriptsize\insertframenumber}}}\hspace*{5pt}}

% Space between paragraphs on notes page
\addtobeamertemplate{note page}{\setlength{\parskip}{12pt}}

% Color and shape of bullets
% \setbeamercolor{item}{fg=gray} 
% \setbeamercolor{subitem}{fg=gray}
% \setbeamercolor{itemize/enumerate subbody}{fg=gray}
\setbeamertemplate{itemize item}{{\textendash}}
\setbeamertemplate{itemize subitem}{{\textendash}}
\setbeamerfont{itemize/enumerate subbody}{size=\footnotesize}
\setbeamerfont{itemize/enumerate subitem}{size=\footnotesize}

\usepackage{amssymb,amsmath}
\usepackage{ifxetex,ifluatex}
\usepackage{fixltx2e} % provides \textsubscript
\ifxetex
  \usepackage{fontspec,xltxtra,xunicode}
  \defaultfontfeatures{Mapping=tex-text,Scale=MatchLowercase}
\else
  \ifluatex
    \usepackage{fontspec}
    \defaultfontfeatures{Mapping=tex-text,Scale=MatchLowercase}
  \else
    \usepackage[utf8]{inputenc}
  \fi
\fi
\usepackage{graphicx}
% Redefine \includegraphics so that, unless explicit options are
% given, the image width will not exceed the width of the page.
% Images get their normal width if they fit onto the page, but
% are scaled down if they would overflow the margins.
\makeatletter
\makeatother
\let\Oldincludegraphics\includegraphics
\renewcommand{\includegraphics}[2][]{\Oldincludegraphics[width=\textwidth,height=0.7\textheight,keepaspectratio]{#2}}

% Comment these out if you don't want a slide with just the
% part/section/subsection/subsubsection title:
%\AtBeginPart{
%  \let\insertpartnumber\relax
%  \let\partname\relax
%  \frame{\partpage}
%}
%\AtBeginSection{
%  \let\insertsectionnumber\relax
%  \let\sectionname\relax
%  \frame{\sectionpage}
%}
%\AtBeginSubsection{
%  \let\insertsubsectionnumber\relax
%  \let\subsectionname\relax
%  \frame{\subsectionpage}
%}

\setlength{\parindent}{0pt}
\setlength{\parskip}{6pt plus 2pt minus 1pt}
\setlength{\emergencystretch}{3em}  % prevent overfull lines
\setcounter{secnumdepth}{0}
\usepackage{pgfpages}
\pgfpagesuselayout{2 on 1}
\providecommand{\tightlist}{%
\setlength{\itemsep}{0pt}\setlength{\parskip}{0pt}}
\makeatletter
\makeatother
\let\Oldincludegraphics\includegraphics
\renewcommand{\includegraphics}[2][]{\Oldincludegraphics[width=\textwidth,height=0.7\textheight,keepaspectratio]{#2}}
\newcommand{\highlight}[1]{%
  \colorbox{red!50}{$\displaystyle#1$}}

\begin{document}

\begin{frame}{Continuum Mechanics}
\protect\hypertarget{continuum-mechanics}{}
Lecture 13 - Anisotropic Hyperelasticity

Dr.~Nicholas Smith

Wichita State University, Department of Aerospace Engineering

29 October, 2020
\end{frame}

\begin{frame}{schedule}
\protect\hypertarget{schedule}{}
\begin{itemize}
\tightlist
\item
  29 Oct - Newtonian Fluids
\item
  3 Nov - Newtonian Fluids
\item
  5 Nov - Reynolds Transport Theorem
\item
  10 Nov - Viscoelastic Materials
\item
  12 Nov - Viscoelastic Materials
\end{itemize}
\end{frame}

\begin{frame}{outline}
\protect\hypertarget{outline}{}
\begin{itemize}
\tightlist
\item
  newtonian fluids
\item
  flow conditions
\end{itemize}
\end{frame}

\begin{frame}{fluids in rigid motion}
\protect\hypertarget{fluids-in-rigid-motion}{}
\begin{itemize}
\item
  We define a fluid as a material which is unable to resist shear stress
  at rest
\item
  For a fluid in rigid body motion, the stress vector on any plane will
  be normal to that plane \[T_{ij}n_j = \lambda n_j\]
\item
  The symmetry of the stress tensor leads us to find that
  \[T_{ij} = -p \delta_{ij}\]
\end{itemize}
\end{frame}

\begin{frame}{compressible and incompressible fluids}
\protect\hypertarget{compressible-and-incompressible-fluids}{}
\begin{itemize}
\item
  Most liquids can be treated as incompressible in many fluid problems
\item
  Their change in density is negligible under a wide range of pressures
\item
  Most gases, however, must be treated as compressible
\item
  Recall the conservation of mass
  \[\frac{D}{Dt} \rho + \rho \frac{\partial v_k}{\partial x_k} = 0\]
\item
  Which for an incompressible material becomes
  \[\frac{\partial v_k}{\partial x_k} = 0\]
\item
  Density of an incompressible material can vary in space, as long as it
  does not vary in time
\end{itemize}
\end{frame}

\begin{frame}{hydrostatics}
\protect\hypertarget{hydrostatics}{}
\begin{itemize}
\item
  If we substitute \(T_{ij} = -p \delta_{ij}\) into the equilibrium
  equations, we find \[\frac{\partial p}{\partial x_i} = \rho B_i\]
\item
  If gravity is the only body force and acts in \(x_3\), then pressure
  will only be a function of \(x_3\) (for static fluid)
\item
  If the fluid is in rigid body motion then we have
  \[-\frac{\partial p}{\partial x_i} + \rho B_i = \rho a_i\]
\end{itemize}
\end{frame}

\begin{frame}{example}
\protect\hypertarget{example}{}
\begin{itemize}
\tightlist
\item
  You are planning to load your fish tank into your friend's car for
  transportation
\item
  Your friend brags that he can accelerate from 0 to 60 in 5 seconds
\item
  Assuming this is true, and your tank is 2'x4' and 2' deep, how deep
  can you fill the tank without allowing any spilling due to
  acceleration?
\end{itemize}
\end{frame}

\begin{frame}{general motion of fluids}
\protect\hypertarget{general-motion-of-fluids}{}
\begin{itemize}
\item
  For a fluid in general motion, we de-compose the stress tensor into
  two portions \[T_{ij} = -p \delta_{ij} + T_{ij}^\prime\]
\item
  Where \(T_{ij}^\prime\) depends only on the rate of deformation and
  \(p\) is a scalar which does not depend on the rate of deformation
\end{itemize}
\end{frame}

\begin{frame}{newtonian fluids}
\protect\hypertarget{newtonian-fluids}{}
\begin{itemize}
\tightlist
\item
  For a fluid to be Newtonian, we make two assumptions
\item
  First, we assume that \(T_{ij}^\prime\) is linearly dependent on
  \(D_{ij}\) and nothing else
\item
  Second, we assume the fluid is isotropic
\item
  This gives \[T_{ij}^\prime = \lambda D_{kk}\delta_{ij} + 2\mu D_{ij}\]
\end{itemize}
\end{frame}

\begin{frame}{physical interpretation}
\protect\hypertarget{physical-interpretation}{}
\begin{itemize}
\item
  If we consider a shear flow given by the velocity field
  \[v_1 = f(x_2) \qquad v_2 = v_3 = 0\]
\item
  We have a rate of deformation tensor with
  \[D_{12} = \frac{1}{2} \frac{d v_1}{d x_2}\]
\item
  With all other \(D_{ij} = 0\)
\item
  Thus we find \(T_{12} = \mu \frac{dv_1}{dx_2}\)
\item
  \(\mu\) relates shear stress to the rate of change of the angle, is
  known as viscosity
\end{itemize}
\end{frame}

\begin{frame}{physical interpretation}
\protect\hypertarget{physical-interpretation-1}{}
\begin{itemize}
\item
  For a general velocity field, if we take \(1/3\) of the contraction of
  the viscous stress tensor, we find
  \[\frac{1}{3} T_{ii}^\prime = \left(\lambda + \frac{2\mu}{3}\right) D_{ii}\]
\item
  The quantity \(\left(\lambda + \frac{2\mu}{3}\right)\) relates the
  mean viscous normal stress to the change in volume
\item
  It is often referred to as the bulk viscosity
\end{itemize}
\end{frame}

\begin{frame}{incompressible fluid}
\protect\hypertarget{incompressible-fluid}{}
\begin{itemize}
\item
  If a fluid is considered to be incompressible, then \(D_{ii} = 0\)
\item
  This gives the constitutive equation
  \[T_{ij} = -p \delta_{ij} + 2\mu D_{ij}\]
\item
  It is convenient to write it in terms of the velocity vector
  \[T_{ij} = -p \delta_{ij} + 2\mu (v_{i,j} + v_{j,i})\]
\end{itemize}
\end{frame}

\begin{frame}{navier-stokes}
\protect\hypertarget{navier-stokes}{}
\begin{itemize}
\item
  If we recall Navier-Stokes equations of motion
  \[\rho \left ( \frac{\partial v_i}{\partial t} + v_j \frac{\partial v_i}{\partial x_j}\right) = \frac{\partial T_{ij}}{\partial x_j} + \rho B_i\]
\item
  We can substitute the constitutive equation for newtonian fluids to
  find
  \[\rho \left ( \frac{\partial v_i}{\partial t} + v_j \frac{\partial v_i}{\partial x_j}\right) = \rho B_i - \frac{\partial p}{\partial x_i} + \mu \frac{\partial ^2 v_i}{\partial x_j \partial x_j}\]
\item
  This gives three equations with four unknowns, we use the continuity
  equation to find the fourth unknown
  \[\frac{\partial v_i}{\partial x_i} = 0\]
\end{itemize}
\end{frame}

\begin{frame}{cylindrical and spherical coordinates}
\protect\hypertarget{cylindrical-and-spherical-coordinates}{}
\begin{itemize}
\tightlist
\item
  Navier-Stokes equations in cylindrical and spherical coordinates are
  found on p.~364-365 of the text
\item
  There is a typo in 6.8.1, should read \[\begin{gathered}
    \frac{\partial v_r}{\partial r} + v_r \frac{\partial v_r}{\partial r} + \frac{v_\theta}{r} \left(\frac{\partial v_r}{\partial \theta} - v_\theta\right) + v_z \frac{\partial v_r}{\partial z} = -\frac{1}{\rho} \frac{\partial p}{\partial r} + B_r\\
    + \frac{\mu}{\rho} \left[\frac{\partial ^2 v_r}{\partial r^2} + \frac{1}{r^2}\frac{\partial^2v_r}{\partial \theta^2} + \frac{\partial^2v_r}{\partial z^2} + \frac{1}{r}\frac{\partial v_r}{\partial r} - \frac{2}{r^2}\frac{\partial v_\theta}{\partial \theta} - \frac{v_r}{r^2}\right]
  \end{gathered}\]
\end{itemize}
\end{frame}

\begin{frame}{nonslip}
\protect\hypertarget{nonslip}{}
\begin{itemize}
\tightlist
\item
  A common assumption is that of \emph{nonslip} boundaries
\item
  Agrees well with experiments
\item
  Both Newtonian and non-Newtonian fluids
\item
  Fluid moves with boundary, for rigid boundaries the velocity at the
  boundary is 0
\end{itemize}
\end{frame}

\begin{frame}{streamline}
\protect\hypertarget{streamline}{}
\begin{itemize}
\tightlist
\item
  In general, fluid flow is characterized by a velocity field
\item
  As a vector field, there are different ways in which to visualize the
  field
\item
  Streamlines, pathlines, streaklines and timelines are common ways we
  talk about fluids
\end{itemize}
\end{frame}

\begin{frame}{steady and unsteady flow}
\protect\hypertarget{steady-and-unsteady-flow}{}
\begin{itemize}
\tightlist
\item
  A flow is called \emph{steady} if it is fixed in time (at a fixed
  location)
\item
  Otherwise it is called unsteady
\item
  Steady flow does not mean the material derivative is zero
  (\(D\Psi/Dt \ne 0\))
\item
  But it does mean that the partial derivative with respect to time is
  zero (\(\partial \Psi / \partial t = 0\))
\item
  For steady flow, streamlines, streaklines, and pathlines are the same
\end{itemize}
\end{frame}

\begin{frame}{streamline}
\protect\hypertarget{streamline-1}{}
\begin{itemize}
\tightlist
\item
  A streamline is a curve which is instantaneously tangent to the
  velocity vector
\item
  Experimentally, streamlines can be found on the surface of a fluid by
  sprinkling reflective particles and making a short-time exposure
  photograph
\item
  Mathematically, streamlines can be found by considering a parametric
  equation for a curve \(x_i = x_i(s)\)
\item
  We choose \(s\) so that \(dx_i/ds = v_i\) and \(s=0\) corresponds to
  the point \(x_0\), which is the originating point of our streamline
\end{itemize}
\end{frame}

\begin{frame}{streamline example}
\protect\hypertarget{streamline-example}{}
\begin{itemize}
\tightlist
\item
  Given the velocity field
  \[v_i = \langle \frac{kx_1}{1+\alpha t}, kx_2, 0 \rangle\]
\end{itemize}

find the streamline passing through \((a_1,a_2,a_3)\) at time \(t\)
\end{frame}

\begin{frame}{pathline}
\protect\hypertarget{pathline}{}
\begin{itemize}
\tightlist
\item
  A pathline is the path traversed by a fluid particle
\item
  Experimentally, pathlines can be found by using one reflective
  particle and a long-time exposure photograph
\item
  Mathematically, the pathline can be obtained from the velocity field
  as follows \[\begin{aligned}
    \frac{dx_i}{dt} &= v_i(x_i,t)\\
    x_i(t_0) = X_i
  \end{aligned}\]
\end{itemize}
\end{frame}

\begin{frame}{pathline example}
\protect\hypertarget{pathline-example}{}
\begin{itemize}
\tightlist
\item
  Given the velocity field
  \[v_i = \langle \frac{kx_1}{1+\alpha t}, kx_2, 0 \rangle\]
\end{itemize}

find the pathline passing through \((a_1,a_2,a_3)\) at time \(t\)
\end{frame}

\begin{frame}{streakline}
\protect\hypertarget{streakline}{}
\begin{itemize}
\tightlist
\item
  Streaklines are commonly found experimentally, but are difficult to
  express mathematically
\item
  A streakline is formed when dye is steadily injected into a fluid from
  a fixed point
\item
  The path that the very first point of dye follows is a pathline
\item
  But the dye following behind is altered by the changing flow field,
  which makes the streakline left by the continuously injected dye
  different from a pathline
\end{itemize}
\end{frame}

\begin{frame}{timeline}
\protect\hypertarget{timeline}{}
\begin{itemize}
\tightlist
\item
  The final common method for visualizing fluid flows is known as a
  timeline
\item
  Fluid particles are marked at a given instance of time (often forming
  a line at \(t_0\))
\item
  After set intervals of time, lines are drawn between these particles
\item
  These lines are called timelines
\end{itemize}
\end{frame}

\begin{frame}{animation}
\protect\hypertarget{animation}{}
\includegraphics{../images/Streaklines_and_pathlines_animation.gif}
\end{frame}

\begin{frame}{laminar flow}
\protect\hypertarget{laminar-flow}{}
\begin{itemize}
\tightlist
\item
  Laminar flow is very orderly
\item
  Fluid particles move in smooth layers (\emph{laminae})
\item
  Occurs when fluid flow is relatively slow
\end{itemize}
\end{frame}

\begin{frame}{reynolds number}
\protect\hypertarget{reynolds-number}{}
\begin{itemize}
\item
  Dimensionless parameter to compare how ``fast'' or ``slow'' a fluid is
  moving
\item
  For experiments under otherwise identical conditions, reynolds number
  is used to determine whether flow will be laminar
\item
  Ratio of inertial forces to viscous forces
\item
  In a tube, Reynolds number is \[N_R = \frac{v_m \rho d}{\mu}\]
\item
  For water in a tube, \(N_R < 2100\) gives laminar flow
\end{itemize}
\end{frame}

\begin{frame}{turbulent flow}
\protect\hypertarget{turbulent-flow}{}
\begin{itemize}
\tightlist
\item
  In laminar flow, small perturbations are quickly overcome
\item
  For turbulent flow, unsteady vortices appear and interact with each
  other
\item
  Turbulent flows are highly irregular and chaotic
\item
  Turbulence increases diffusivity, causing fluids to mix more quickly
\item
  High Reynolds numbers correspond to turbulence, but how high depends
  on the specific experiment
\item
  There is often a large transition range between laminar and turbulent
  flow
\end{itemize}
\end{frame}

\begin{frame}{reading}
\protect\hypertarget{reading}{}
\begin{itemize}
\tightlist
\item
  pp 365-375
\end{itemize}
\end{frame}

\end{document}
