% Options for packages loaded elsewhere
\PassOptionsToPackage{unicode}{hyperref}
\PassOptionsToPackage{hyphens}{url}
%
\documentclass[
  letterpaper,
  ignorenonframetext,
  aspectratio=43,
  handout,
  12pt]{beamer}
\usepackage{pgfpages}
\setbeamertemplate{caption}[numbered]
\setbeamertemplate{caption label separator}{: }
\setbeamercolor{caption name}{fg=normal text.fg}
\beamertemplatenavigationsymbolsempty
% Prevent slide breaks in the middle of a paragraph
\widowpenalties 1 10000
\raggedbottom
\setbeamertemplate{part page}{
  \centering
  \begin{beamercolorbox}[sep=16pt,center]{part title}
    \usebeamerfont{part title}\insertpart\par
  \end{beamercolorbox}
}
\setbeamertemplate{section page}{
  \centering
  \begin{beamercolorbox}[sep=12pt,center]{part title}
    \usebeamerfont{section title}\insertsection\par
  \end{beamercolorbox}
}
\setbeamertemplate{subsection page}{
  \centering
  \begin{beamercolorbox}[sep=8pt,center]{part title}
    \usebeamerfont{subsection title}\insertsubsection\par
  \end{beamercolorbox}
}
\AtBeginPart{
  \frame{\partpage}
}
\AtBeginSection{
  \ifbibliography
  \else
    \frame{\sectionpage}
  \fi
}
\AtBeginSubsection{
  \frame{\subsectionpage}
}
\usepackage{lmodern}
\usepackage{amssymb,amsmath}
\usepackage{ifxetex,ifluatex}
\ifnum 0\ifxetex 1\fi\ifluatex 1\fi=0 % if pdftex
  \usepackage[T1]{fontenc}
  \usepackage[utf8]{inputenc}
  \usepackage{textcomp} % provide euro and other symbols
\else % if luatex or xetex
  \usepackage{unicode-math}
  \defaultfontfeatures{Scale=MatchLowercase}
  \defaultfontfeatures[\rmfamily]{Ligatures=TeX,Scale=1}
\fi
\usetheme[]{metropolis}
% Use upquote if available, for straight quotes in verbatim environments
\IfFileExists{upquote.sty}{\usepackage{upquote}}{}
\IfFileExists{microtype.sty}{% use microtype if available
  \usepackage[]{microtype}
  \UseMicrotypeSet[protrusion]{basicmath} % disable protrusion for tt fonts
}{}
\makeatletter
\@ifundefined{KOMAClassName}{% if non-KOMA class
  \IfFileExists{parskip.sty}{%
    \usepackage{parskip}
  }{% else
    \setlength{\parindent}{0pt}
    \setlength{\parskip}{6pt plus 2pt minus 1pt}}
}{% if KOMA class
  \KOMAoptions{parskip=half}}
\makeatother
\usepackage{xcolor}
\IfFileExists{xurl.sty}{\usepackage{xurl}}{} % add URL line breaks if available
\IfFileExists{bookmark.sty}{\usepackage{bookmark}}{\usepackage{hyperref}}
\hypersetup{
  hidelinks,
  pdfcreator={LaTeX via pandoc}}
\urlstyle{same} % disable monospaced font for URLs
\newif\ifbibliography
\setlength{\emergencystretch}{3em} % prevent overfull lines
\providecommand{\tightlist}{%
  \setlength{\itemsep}{0pt}\setlength{\parskip}{0pt}}
\setcounter{secnumdepth}{-\maxdimen} % remove section numbering
\usepackage{pgfpages}
\pgfpagesuselayout{2 on 1}
\providecommand{\tightlist}{%
\setlength{\itemsep}{0pt}\setlength{\parskip}{0pt}}
\makeatletter
\makeatother
\let\Oldincludegraphics\includegraphics
\renewcommand{\includegraphics}[2][]{\Oldincludegraphics[width=\textwidth,height=0.7\textheight,keepaspectratio]{#2}}
\ifluatex
  \usepackage{selnolig}  % disable illegal ligatures
\fi

\author{}
\date{}

\begin{document}

\begin{frame}{Continuum Mechanics}
\protect\hypertarget{continuum-mechanics}{}
Lecture 12 - Large Deformation

Dr.~Nicholas Smith

Wichita State University, Department of Aerospace Engineering

8 October, 2020
\end{frame}

\begin{frame}{schedule}
\protect\hypertarget{schedule}{}
\begin{itemize}
\tightlist
\item
  8 Oct - Large Deformation
\item
  13 Oct - Anisotropy and Large Deformation
\item
  15 Oct - Exam Review
\item
  20 Oct - Exam 2
\end{itemize}
\end{frame}

\begin{frame}{outline}
\protect\hypertarget{outline}{}
\begin{itemize}
\tightlist
\item
  anisotropic solution techniques
\item
  large deformation
\item
  simple deformation modes
\end{itemize}
\end{frame}

\begin{frame}{planar problems}
\protect\hypertarget{planar-problems}{}
\begin{itemize}
\tightlist
\item
  Many of our usual solution techniques are more difficult to apply to
  anisotropic materials
\item
  For a general anisotropic material under the assumption of plane
  strain (\(u_i = \langle u_1(x_1,x_2), u_2(x_1,x_2)\rangle\)) the
  equilibrium equations (in terms of displacement) are \[\begin{aligned}
    C_{11} u_{1,11} + C_{66} u_{1,22} + C_{16} (2u_{1,12} + u_{2,11}) + C_{26} u_{2,22} + (C_{12}+C_{66})u_{2,12} &=0\\
    C_{16} u_{1,11} + C_{26} u_{1,22} + (C_{66} + C_{12}) u_{1,12} + C_{66} u_{2,11} + C_{22}u_{2,22} + 2C_{26}u_{2,12} &=0\\
    C_{15} u_{1,11} + C_{46} u_{1,22} + (C_{56} + C_{14})u_{1,12} + C_{56}u_{2,11} + C_{24} u_{2,22} + (C_{25}+C_{46})u_{2,12} &=0
  \end{aligned}\]
\end{itemize}
\end{frame}

\begin{frame}{planar problems}
\protect\hypertarget{planar-problems-1}{}
\begin{itemize}
\tightlist
\item
  Only two of these three equations can be solved with a plane strain
  displacement assumption
\item
  This means that a general anisotropic body cannot be solved in plane
  strain
\item
  However if a material possesses at least monoclinic symmetry, enough
  terms vanish that plane strain is an acceptable solution
\item
  Alternatively, a generalized plane strain solution can be used for any
  anisotropic body
  \[u = \langle u_1(x_1,x_2), u_2(x_1,x_2), u_3(x_1,x_2) \rangle\]
\end{itemize}
\end{frame}

\begin{frame}{stroh representation}
\protect\hypertarget{stroh-representation}{}
\begin{itemize}
\tightlist
\item
  Stroh developed a complex variable representation for generalized
  plane strain solutions in anisotropic materials
\item
  If we assume a displacement field in the form \(u_i = a_i f(z)\) where
  \(z = x_1 + px_2\) the equilibrium equations in terms of displacement
  then become \[C_{ijkl} u_{k,il} = 0\]
\end{itemize}
\end{frame}

\begin{frame}{stroh representation}
\protect\hypertarget{stroh-representation-1}{}
\begin{itemize}
\tightlist
\item
  Since \(\partial z/ \partial x_i = \delta_{i1} + p \delta_{i2}\), we
  find that
  \[u_{k,l} = a_k (\delta_{l1} + p \delta_{l2})f^\prime(z) \qquad u_{k,il} = a_k(\delta_{l1 + p\delta_{l2}})(\delta_{j1} + p\delta_{j2})f^{\prime\prime}(z)\]
\end{itemize}
\end{frame}

\begin{frame}{stroh representation}
\protect\hypertarget{stroh-representation-2}{}
\begin{itemize}
\item
  The equilibrium equations with a generalized plane strain displacement
  function now become
  \[C_{ijkl}(\delta_{l1} + p\delta_{l2})(\delta_{j1} + p\delta_{j2})a_kf^{\prime \prime}(z) = 0\]
\item
  For non-trivial solutions (when \(f^{\prime \prime}(z) \ne 0\)), we
  can re-write the equations as \[a_k = 0\]
\end{itemize}
\end{frame}

\begin{frame}{stroh representation}
\protect\hypertarget{stroh-representation-3}{}
\begin{itemize}
\tightlist
\item
  If we define the following quantities in terms of the stiffness tensor
  \[Q = \begin{bmatrix}
    C_{11} & C_{16} & C_{15}\\
    C_{16} & C_{66} & C_{56}\\
    C_{15} & C_{56} & C_{55}
  \end{bmatrix} \qquad R = \begin{bmatrix}
    C_{16} & C_{12} & C_{14}\\
    C_{66} & C_{26} & C_{46}\\
    C_{56} & C_{25} & C_{45}
  \end{bmatrix} \qquad T = \begin{bmatrix}
    C_{66} & C_{26} & C_{46}\\
    C_{26} & C_{22} & C_{24}\\
    C_{46} & C_{24} & C_{44}
  \end{bmatrix}\]
\end{itemize}
\end{frame}

\begin{frame}{stroh representation}
\protect\hypertarget{stroh-representation-4}{}
\begin{itemize}
\item
  we can now re-write the equilibrium equations in matrix form as
  \[a=0\]
\item
  This is a type of eigenvalue problem
\end{itemize}
\end{frame}

\begin{frame}{stroh representation}
\protect\hypertarget{stroh-representation-5}{}
\begin{itemize}
\tightlist
\item
  Mathematically, it can be shown that for a material with physically
  admissible elastic constants, \(p\) will always be complex
\item
  There will be three pairs of solutions \((p,\bar{p})\), and three
  pairs of complex-valued eigenvectors, \((a_i,\bar{a_i})\)
\end{itemize}
\end{frame}

\begin{frame}{stroh representation}
\protect\hypertarget{stroh-representation-6}{}
\begin{itemize}
\item
  Stress solutions can be calculated by defining \(b_i\) vectors
  \[a^{(i)} = b^{(i)}\]
\item
  Which gives stress solutions as
  \[\sigma_{i1} = -pb_if^\prime(z) \qquad \sigma_{i2} = b_i f^\prime(z)\]
\end{itemize}
\end{frame}

\begin{frame}{change of frame}
\protect\hypertarget{change-of-frame}{}
\begin{itemize}
\tightlist
\item
  For small deformations, the current and deformed frame have negligible
  differences
\item
  For large deformations, we need to ensure that our constitutive law is
  objective, or frame-indifferent
\end{itemize}
\end{frame}

\begin{frame}{change of frame}
\protect\hypertarget{change-of-frame-1}{}
\begin{itemize}
\tightlist
\item
  Examples:

  \begin{itemize}
  \tightlist
  \item
    distance between two material points is a frame-indifferent scalar
  \item
    speed of a material point is not frame-independent (non-objective)
  \item
    position vector and velocity vector of a material point are
    non-objective
  \item
    relative velocity between two material points is objective
  \end{itemize}
\end{itemize}
\end{frame}

\begin{frame}{change of frame}
\protect\hypertarget{change-of-frame-2}{}
\begin{itemize}
\tightlist
\item
  In general, a ``frame'' can have its own time scale, origin, and
  directionality
\item
  a change of frame would then be given by
  \[x_i^* = c_i(t) + Q_{ij}(t)(x_j-x^{0}_j)\]
\end{itemize}
\end{frame}

\begin{frame}{change of frame}
\protect\hypertarget{change-of-frame-3}{}
\begin{itemize}
\item
  If we consider the position vector of two material points, in the
  starred frame we have
  \[x_1^* = c(t) + Q(t)(x_1-x_0) \qquad x_2^* = c(t) + Q(t)(x_2-x_0)\]
\item
  The relative position vector, \(b=x_1-x_2\) can also be found in the
  starred frame as \[b^* = x_1^*-x_2^* = Q(t)(x_1-x_2)\]
\item
  Any vector which obeys this law is known as an objective vector
\end{itemize}
\end{frame}

\begin{frame}{change of frame}
\protect\hypertarget{change-of-frame-4}{}
\begin{itemize}
\item
  If we consider some tensor, \(T_{ij}\) which transforms an objective
  vector \(b_j\) into another objective vector, \(c_i\)
  \[c_i = T_{ij}b_j\]
\item
  Since both \(b\) and \(c\) are objective vectors, we can write
  \[c_i^* = Q_{ik} c_k = Q_{ik} T_{kj} b_j = Q_{ik} T_{kj} Q_{lj} b_l\]
\end{itemize}
\end{frame}

\begin{frame}{change of frame}
\protect\hypertarget{change-of-frame-5}{}
\begin{itemize}
\tightlist
\item
  This means that for \(T_{ij}\) to be objective it must satisfy
  \[T_{ij}^* = Q_{ik} T_{kl} Q_{jl}\]
\end{itemize}
\end{frame}

\begin{frame}{examples}
\protect\hypertarget{examples}{}
\begin{itemize}
\tightlist
\item
  \(dx_i\)
\item
  \(ds\)
\item
  \(v_i\)
\item
  \(F_{ij}\)
\item
  \(C_{ij}\)
\item
  \(B_{ij}\)
\end{itemize}
\end{frame}

\begin{frame}{group problems}
\protect\hypertarget{group-problems}{}
\begin{itemize}
\tightlist
\item
  Group 1: Is the first Piola-Kirchhoff stress tensor objective? How
  does it transform? Recall: \[T_{ij}^0 = J T_{ik} F_{jk}^{-1}\]
\item
  Group 2: Is the second Piola-Kirchhoff stress tensor objective? How
  does it transform? Recall:
  \[\tilde{T_{ij}} = J F_{ik}^{-1} T_{kl} F_{jl}^{-1}\]
\end{itemize}
\end{frame}

\begin{frame}{constitutive equations}
\protect\hypertarget{constitutive-equations}{}
\begin{itemize}
\tightlist
\item
  For large deformation, a constitutive law must be objective
\item
  \(T_{ij} = f(C_{ij})\) is not an acceptable form, but
  \(T_{ij} = f(B_{ij})\) is
\item
  \(\tilde{T}_{ij} = f(C_{ij})\) is also an acceptable form of the
  constitutive equation
\item
  If we assume our material to be isotropic, it can be shown that, with
  no loss of generality
  \[f(B_{ij}) = a_0 \delta_{ij} + a_1 B_{ij} + a_2 B_{ij}^2\]
\end{itemize}
\end{frame}

\begin{frame}{constitutive equations}
\protect\hypertarget{constitutive-equations-1}{}
\begin{itemize}
\item
  A commonly used alternate form of the constitutive equation is
  \[T_{ij} = \varphi_0 \delta_{ij} + \varphi_1 B_{ij} + \varphi_2 B_{ij}^{-1}\]
\item
  If the material is incompressible, the stress is indeterminate to some
  arbitrary hydrostatic pressure
  \[T_{ij} = -p \delta_{ij} + \varphi_1 B_{ij} + \varphi_2 B_{ij}^{-1}\]
\end{itemize}
\end{frame}

\begin{frame}{money rivlin}
\protect\hypertarget{money-rivlin}{}
\begin{itemize}
\tightlist
\item
  This is known as the Mooney-Rivlin model, which can be written in
  different ways
  \[T_{ij} = -p \delta_{ij} + \mu \left(\frac{1}{2} + \beta\right) B_{ij} - \mu \left(\frac{1}{2} - \beta\right) B_{ij}^{-1}\]
\end{itemize}
\end{frame}

\begin{frame}{neo-hookean solid}
\protect\hypertarget{neo-hookean-solid}{}
\begin{itemize}
\tightlist
\item
  A simpler model, which is only accurate for strains less than 20\%, is
  the neo-Hookean solid
\item
  For incompressible materials, the neo-Hookean equation is
  \[T_{ij} = -p \delta_{ij} + 2C_1 B_{ij}\]
\end{itemize}
\end{frame}

\begin{frame}{neo-hookean solid}
\protect\hypertarget{neo-hookean-solid-1}{}
\begin{itemize}
\tightlist
\item
  Where, for consistency with Hooke's Law, \(2 C_1 = \mu\)
\item
  When large tensile strains are not important, the neo-Hookean model is
  popular because it only needs one material constant, which has more
  physical meaning than Mooney-Rivlin constants.
\end{itemize}
\end{frame}

\begin{frame}{incompressible stretch}
\protect\hypertarget{incompressible-stretch}{}
\begin{itemize}
\item
  In large deformation problems, the stretch ratio, \(\lambda\) is often
  used (instead of strain)
\item
  \(\lambda_1\) represents the ratio of deformed length in \(x_1\) to
  undeformed length in \(x_1\)
\item
  For uniaxial extension we have \[\begin{aligned}
    x_1 &= \lambda_1 X_1\\
    x_2 &= \lambda_2 X_2\\
    x_3 &= \lambda_2 X_3
  \end{aligned}\]
\item
  Also if the material is incompressible we know
  \[\lambda_1 \lambda_2^2 = 1\]
\end{itemize}
\end{frame}

\begin{frame}{simple shear}
\protect\hypertarget{simple-shear}{}
\begin{itemize}
\item
  For large shear deformation, we have \[\begin{aligned}
    x_1 &= X_1 + KX_2\\
    x_2 &= X_2\\
    x_3 &= X_3\\
  \end{aligned}\]
\item
  And we find \(B\) and \(B^{-1}\) as \[B_{ij} = \begin{bmatrix}
    1+K^2 & K & 0\\
    K & 1 & 0\\
    0 & 0 & 1
  \end{bmatrix} \qquad B_{ij}^{-1} = \begin{bmatrix}
    1 & -K & 0\\
    -K & 1+ K^2 & 0\\
    0 & 0 & 1
  \end{bmatrix}\]
\end{itemize}
\end{frame}

\begin{frame}{reading material}
\protect\hypertarget{reading-material}{}
\begin{itemize}
\tightlist
\item
  Thermodynamics formulation of Mooney-Rivlin models -
  \href{http://continuummechanics.org/mooneyrivlin.html}{link}
\item
  Anisotropy in large deformation - ``A New Constitutive Framework for
  Arterial Wall Mechanics and a Comparative Study of Material Models''
\item
  Paper is interesting, but long, pp 10-21 are the most relevant.
\end{itemize}
\end{frame}

\end{document}
